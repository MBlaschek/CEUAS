\begin{table}[!htbp] 
\footnotesize
\begin{center}
\renewcommand{\arraystretch}{1.3}
\begin{tabular}{ l l l p{3.5in}} 
\toprule 
\textbf{Variable} & \textbf{CDM Name} & \textbf{Units} & \textbf{Description}  \\ \toprule \toprule
date &  & [] & \ 
codetype &  & [] & \ 
time &  & [] & \ 
vertco_reference_1 & ??? & [] & ???\ 
lat & latitude & [] & Latitude of station, -90 to 90 (or other as defined by station_crs\ 
vertco_type & z_coordinate_type & [] & Type of z coordinate\ 
stalt & observation_height_above_station_surface & [] & Height of sensor above local ground or sea surface. Positive values for above surface (e.g. sondes), negative for below (e.g. xbt). For visualobservations, height of the visual observing platform.\ 
long & longitude & [] & Longitude of station, -180.0 to 180 (or others as defined by station_crs)\ 
obstype &  & [] & \ 
\bottomrule \\bottomrule
\end{tabular}
\end{center}
\caption{Definition of naming convention, description and units for the variables contained in the netCDF files.}
\label{CDM}
\end{table}
