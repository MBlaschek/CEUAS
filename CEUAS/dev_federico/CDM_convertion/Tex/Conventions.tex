\documentclass[a4paper,11pt]{article}
%\pdfoutput=1 % if your are submitting a pdflatex (i.e. if you have
             % images in pdf, png or jpg format)
\usepackage{jheppub} % for details on the use of the package, please
                     % see the JHEP-author-manual
\usepackage[T1]{fontenc} % if needed

\usepackage{slashed}
%\usepackage{subfigure}
\usepackage{xspace}
\usepackage{booktabs}
\author[a]{Federico Ambrogi}

% e-mail addresses: one for each author, in the same order as the authors\emailAdd{federico.ambrogi@oeaw.ac.at}
\emailAdd{federico.ambrogi88@gmail.com}

\newcommand{\dp}{\textit{dp~}}
\newcommand{\rh}{\textit{rh~}}
%% %simple case: 2 authors, same institution
%% \author{A. Uthor}
%% \author{and A. Nother Author}
%% \affiliation{Institution,\\Address, Country}


\title{\boldmath Summary of Conventions}

\begin{document} 
	\sffamily
	\maketitle
	
	
	
	
\section{Convention Tables}
Tabble \ref{CDM} summarises the naming convention for the variables used in netCDF files (converted from the odb files with the \textit{readodbfile\_station.py} script).
The variable definition, their naming convention, and the physics units follow the preliminary CDM-common data model agreement, that can be found at:
\newline
\url{https://github.com/glamod/common_data_model/}
\newline
\url{https://github.com/glamod/common_data_model/blob/master/tables/observed_variable.dat}



%\input(a.tex) does not work on windows?


\begin{table}[!htbp]
	\footnotesize
	\begin{center}
		\renewcommand{\arraystretch}{1.3}
		\begin{tabular}{ l l l p{3.5in}}
			\toprule
			\textbf{Variable} & \textbf{CDM Name} & \textbf{Units} & \textbf{Description}  \\ \toprule \toprule
			pressure & pressure & [Pa] & pressure of air column at specified height\\
			dewpoint & dew point temperature & [K] & Dew point temperature is the temperature at which a parcel of air reaches saturation upon being cooled at constant pressure and specific humidity.\\
			wind & wind & [m s-1] & Speed is the magnitude of velocity. Wind is defined as a two-dimensional (horizontal) air velocity vector,  with no vertical component. (Vertical motion in the atmosphere has the standard name upward air velocity.) The wind speed is the magnitude of the wind velocity. Lot 1 uses ff  - WMO abbrev.\\
			humidity & specific humidity & [g kg-1] & specific means per unit mass. Specific humidity is the mass fraction of water vapor in (moist) air.\\
			\bottomrule \bottomrule
		\end{tabular}
	\end{center}
	\caption{Definition of naming convention, description and units for the variables contained in the netCDF files.}
	\label{CDM}
\end{table}

\section{Relative Humidity and Dew Point Temperature}
In this section the procedure to extract the relative humidity (\rh) and dew point temperature (\dp) is described.
The measured data for the rh and dp might be contained in the original odb file. In this case, as for the other variables such as air temperature and wind speed, the data is extracted directly from the odb files and converted to the netCDF format.
However the \rh or the \dp can be calculated from the other variable if the temperature is also known, using the saturation water vapor pressure. The formula known as \textit{FOEEWMO} (see the humidity.py script) gives the saturation water vapor pressure in Pascal:
\begin{equation}
FOEEWMO = 611.21 \times e^{17.502 \times \frac{t - 273.16}{t -32.19} } .
\end{equation}
The \rh temperature can be calculated as:
\begin{equation}
rh = \frac{FOEEWMO(dp)}{FOEEWMO(t)}
\end{equation}
i.e. evaluating the ratio of the saturation water vapor pressure at dew point temperature and at the measured air temperature. For the station 10939, which often reports directly both the \rh and \dp values, it was tested that the \rh measure values matched the values calculated using the formula above \footnote{Note that in the humidity.py script, the formula sh2rh allows to calculate the \rh from the specific humidity; however a discrepancy of constant factor $\sim$0.62 was found between the calculated and observed values.}.
\\
It is also possible to calculate the \dp knowing the \rh value by inverting the FOEEWMO formula as:
\begin{equation}
FOEEWMO(dp) = FOEEWMO(t) \times rh 
\end{equation}

\begin{equation*}
611.21 \times e^{17.502 \frac{dp - 273.16}{dp-32.19}} = FOEEWMO(t) \times rh
\end{equation*}

\begin{equation*}
 e^{17.502 \frac{dp - 273.16}{dp-32.19}} =  \frac{FOEEWMO(t) \times rh}{611.21}
\end{equation*}


\begin{equation*}
def \ F(t) \equiv \frac{FOEEWMO(t) \times rh}{611.21}
\end{equation*}


\begin{equation*}
 \frac{dp - 273.16}{dp-32.19} = \frac{ln \left( F(t) \right)}{17.502}   
\end{equation*}

\begin{equation*}
dp - 273.16 = \left(\frac{ln \left(  F(t) \right)}{17.502}   \right) \times dp - \frac{32.19}{17.502} \times ln \left(  F(t) \right)
\end{equation*} 

\begin{equation*}
dp - \left(\frac{ln \left(  F(t) \right)}{17.502}   \right) \times dp  = - \frac{32.19}{17.502} \times ln \left(  F(t) \right) + 273.16
\end{equation*} 


\begin{equation*}
\left (  1 - \frac{ ln \left(  F(t)  \right)}{17.502} \right) \times dp = - 1.839 \times ln \left(   F(t)    \right) + 273.16 
\end{equation*} 




\begin{equation}
dp =  \frac{ - 1.839 \times ln \left(   F(t)    \right) + 273.16 }{\left (  1 - \frac{ ln \left(  F(t)  \right)}{17.502} \right)}
\end{equation}


% The following file contains the tables
% produced automatically by the CDM_Format.py file 



%\input(Tables.txt)

%\bibliographystyle{JHEP}
%\bibliography{references}


\end{document}



