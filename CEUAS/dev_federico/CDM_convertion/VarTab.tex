\begin{table}[!htbp] 
\footnotesize
\begin{center}
\renewcommand{\arraystretch}{1.3}
\begin{tabular}{ l l l p{3.5in}} 
\toprule 
\multicolumn{4}[c][ Variables Table ] \\\textbf{Variable} & \textbf{CDM Name} & \textbf{Units} & \textbf{Description}  \\ \toprule \topruledewpoint & dew point temperature & [K] & Dew point temperature is the temperature at which a parcel of air reaches saturation upon being cooled at constant pressure and specific humidity.\\ 
humidity & specific humidity & [g kg-1] & specific means per unit mass. Specific humidity is the mass fraction of water vapor in (moist) air.\\ 
pressure & pressure & [Pa] & pressure of air column at specified height\\ 
varno & observed_variable & [(int)] & The variable being observed / measured.\\ 
obsvalue & observation_value & [(numeric)] & The observed value.\\ 
wind & wind & [m s-1] & Speed is the magnitude of velocity. Wind is defined as a two-dimensional (horizontal) air velocity vector,  with no vertical component. (Vertical motion in the atmosphere has the standard name upward air velocity.) The wind speed is the magnitude of the wind velocity. Lot 1 uses ff  - WMO abbrev.\\ 
\bottomrule \\bottomrule
\end{tabular}
\end{center}
\caption{Definition of naming convention, description and units for the variables contained in the netCDF files.}
\label{CDM}
\end{table}
