\begin{table}[!htbp] 
\footnotesize
\begin{center}
\renewcommand{\arraystretch}{1.3}
\begin{tabular}{  l p{1.5in} l p{3.0in} } \\
%\multicolumn{4}{c}{ Observation Table } \\
\textbf{Variable} & \textbf{CDM Name} & \textbf{Units} & \textbf{Description}  \\ \toprule
date &  & $[$ $]$ & \\ 
codetype &  & $[$ $]$ & \\ 
time &  & $[$ $]$ & \\ 
vertco\_reference\_1 & ??? & $[$ $]$ & ???\\ 
lat & latitude & $[$ $]$ & Latitude of station, -90 to 90 (or other as defined by station\_crs\\ 
vertco\_type & z\_coordinate\_type & $[$ $]$ & Type of z coordinate\\ 
stalt & observation\_height \_above\_station\_surface & $[$ $]$ & Height of sensor above local ground or sea surface. Positive values for above surface (e.g. sondes), negative for below (e.g. xbt). For visualobservations, height of the visual observing platform.\\ 
long & longitude & $[$ $]$ & Longitude of station, -180.0 to 180 (or others as defined by station\_crs)\\ 
obstype &  & $[$ $]$ & \\ 
\bottomrule \bottomrule
\end{tabular}
\end{center}
\caption{Definition of naming convention, description and units for the variables contained in the netCDF files - Observations}
\label{obs}
\end{table}
